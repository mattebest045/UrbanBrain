\documentclass{article}
\usepackage[utf8]{inputenc}
\usepackage{geometry}
\usepackage{xcolor} % Per la gestione dei colori
\usepackage{fontawesome}
\geometry{
	a4paper,
	total={180mm,265mm}, % Allarga leggermente l'area del contenuto
	left=15mm,
	top=15mm,
}
\usepackage{graphicx}
\usepackage{titling}

\title{Schema Logico del database di UrbanBrain}
\author{Alessandro Bertani, Matteo Bestetti}
\date{Ottobre 2022}

\definecolor{light_blue}{RGB}{0, 101, 191}
\definecolor{map_green}{RGB}{0, 164, 0}
\definecolor{map_red}{RGB}{164, 0, 0}
\newcommand{\book}{\textcolor{light_blue}{\faBook}}
\newcommand{\pin}{\textcolor{map_red}{\faMapMarker}}

\usepackage{fancyhdr}
\fancypagestyle{plain}{%  the preset of fancyhdr 
	\fancyhf{} % clear all header and footer fields
	\fancyfoot[R]{\includegraphics[width=1.5cm]{logo_scuro.png}}
	\fancyfoot[L]{\thedate}
	\fancyhead[L]{Description of Assignment}
	\fancyhead[R]{\theauthor}
}
\pagestyle{plain}

\makeatletter
\def\@maketitle{%
	\newpage
	\null
	\vskip 1em%
	\begin{center}%
		\let \footnote \thanks
		{\LARGE \@title \par}%
		\vskip 1em%
		%{\large \@date}%
	\end{center}%
	\par
	\vskip 1em}
\makeatother

\usepackage{lipsum}  
\usepackage{cmbright}

\begin{document}
	
	\maketitle
	
	\noindent\begin{tabular}{@{}ll}
		Studenti & \theauthor\\
		Promotore &  Prof. Stefano Cagnoni\\
	\end{tabular}
	\section*{Glossario}
	\small % Riduce la dimensione del testo
	
	\begin{itemize}
		\item[\book] Il \underline{Campo sottolineato} indica una chiave primaria per quell'entità.
		\item[\book] Il \textbf{campo in grassetto} indica un campo univoco in un'entità.
		\item[\book] Per convenzione indichiamo come \textit{primary key} un campo che inizia con "ID-", mentre indichiamo come \textit{foreign key} un campo che inizia con "id-".
		\item[\faEdit] \textbf{NOTA}: abbiamo deciso di rinominare i campi delle entità Operatore, Cittadino, SuperAdmin come se fossero delle \textit{primary key} per una migliore leggibililità.
		\item[\textcolor{map_green}{\faMap}] \textbf{Campo} (Sensore.Posizione): Attributo composto utilizzato per rappresentare le coordinate geografiche del sensore in gradi decimali (DD). Include due componenti:\\
		\pin\space \textbf{Latitudine}: valore numerico in gradi decimali che rappresenta la distanza del punto dall'equatore. I valori variano tra -90 (Sud) e +90 (Nord). \\
		\pin\space \textbf{Longitudine}: valore numerico in gradi decimali che indica la distanza del punto dal meridiano di Greenwich. I valori variano tra -180 (Ovest) e +180 (Est).
	\end{itemize}
	
	\section*{Tabelle}
	\small % Riduce la dimensione del testo
	
	Utente(\underline{IDUtente}, Nome, Cognome, DataNascita, \textbf{Email}, \textbf{Telefono}, Indirizzo)\\
	Operatore(\underline{IDOperatore, DataInizio, DataFine}, Email, Tipo, Ruolo, Stato, Password)\\
	Cittadino(\underline{IDCittadino}, Stato, DataRegistrazione, Password)\\
	SuperAdmin(\underline{IDSuperAdmin}, Ruolo, \underline{DataAssegnazioneRuolo}, Stato, UltimoAccesso, Password)\\
	Città(\underline{IDCitta}, Nome, Regione)\\
	RisorsaPubblica(\underline{IDRisorsaPubblica}, Nome, Tipo)\\
	SpesaPubblica(\underline{idRisorsaPubblica, idCitta, idOperatore}, Data, Costo, Stato)\\
	Dato(\underline{idSensore}, Data, Valore)\\
	Sensore(\underline{IDSensore, idCitta}, Posizione, Tipo, DataInstallazione, Stato)\\
	Evento(\underline{IDEvento}, Nome, Luogo, NPosti, Descrizione, Data, Stato)\\
	Creazione(\underline{idCitta, idEvento, idOperatore}, Data, Segnalazione)\\
	Partecipazione(\underline{idCitta, idEvento, idCittadino}, DataPartecipazione, Segnalazione)\\
	Feedback(\underline{IDFeedback}, Tipo)\\
	Segnalazione(\underline{IDSegnalazione, idCitta, idFeedback, idCittadino}, Data, Descrizione, Foto)\\
	Log(\underline{IDLog, idUtente, Data}, Descrizione)\\
	
	%\newpage
	
	\section*{Relazioni (FK \faLongArrowRight\space PK)}
	\small % Riduce la dimensione del testo
	
	Operatore.IDOperatore \faLongArrowRight\space Utente.IDUtente \\
	Cittadino.IDCittadino \faLongArrowRight\space Utente.IDUtente \\
	SuperAdmin.idSuperAdmin \faLongArrowRight\space Utente.IDUtente\\
	SpesaPubblica.idCitta \faLongArrowRight\space Città.IDCitta\\
	SpesaPubblica.idRisorsaPubblica \faLongArrowRight\space RisorsaPubblica.IDRisorsaPubblica\\
	SpesaPubblica.idOperatore \faLongArrowRight\space Operatore.IDOperatore\\
	Sensore.idCitta \faLongArrowRight\space Città.IDCitta\\
	Dato.idSensore \faLongArrowRight\space Sensore.IDSensore\\
	Creazione.idCitta \faLongArrowRight\space Città.IDCitta\\
	Creazione.idEvento \faLongArrowRight\space Evento.IDEvento\\
	Creazione.idOperatore \faLongArrowRight\space Operatore.IDOperatore\\
	Partecipazione.idCitta \faLongArrowRight\space Città.IDCitta\\
	Partecipazione.idEvento \faLongArrowRight\space Evento.IDEvento\\
	Partecipazione.idCittadino \faLongArrowRight\space Cittadino.IDCittadino\\
	Segnalazione.idCitta \faLongArrowRight\space Città.IDCitta\\
	Segnalazione.idCittadino \faLongArrowRight\space Cittadino.IDCittadino\\
	Segnalazione.idFeedback \faLongArrowRight\space Feedback.IDFeedback\\
	Log.idUtente \faLongArrowRight\space Utente.IDUtente
	
\end{document}
